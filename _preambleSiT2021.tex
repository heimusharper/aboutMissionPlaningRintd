%\documentclass[twoside,draft]{article}
\documentclass[twoside]{article}
\usepackage[a5paper]{geometry}
   \geometry{papersize={145mm, 200mm}}
   \geometry{left=18mm}
   \geometry{right=18mm}
   \geometry{top=15mm}
   \geometry{bottom=15mm}
\usepackage{indentfirst} % Красная строка
\usepackage{lipsum}  % this package is for creating filler text
\usepackage[english,russian]{babel}
\usepackage[utf8]{inputenc}
\usepackage{graphicx} % для вставки картинок
\usepackage{cmap} % для кодировки шрифтов в pdf
\usepackage{amsmath,amsfonts,amssymb,amscd,euscript} % поддержка русских букв в формулах
\usepackage{subcaption}
\usepackage[tiny]{titlesec} % размер заголовков big medium small tiny
\usepackage[bottom]{footmisc} % чтоб сноски были в конце страницы
\usepackage{epstopdf} % чтоб работали картинки в Windows
\usepackage{setspace}
\usepackage{multirow} % улучшенное форматирование таблиц
\usepackage{longtable} % работа с длинными таблицами
\usepackage[tableposition=top]{caption}
\usepackage{psfrag}
\usepackage{cite}
\usepackage{listings}
\usepackage{lastpage}
\usepackage{verbatim}
\usepackage{rotating} % для поворота таблиц
\usepackage{wrapfig} % для вписывания изображений в текст
\usepackage{booktabs} % To thicken table lines
\usepackage{lscape}

\lstset{extendedchars=true, %Чтобы русские буквы в комментариях были
		inputencoding=utf8,
        belowcaptionskip=0pt,
        texcl}
        
\graphicspath{{img/}}

\usepackage{fancyhdr}
\pagestyle{fancy}
\fancyhf{}
\fancyfoot[C]{\thepage}
\fancyheadoffset{0mm}
\fancyfootoffset{0mm}
\setlength{\headheight}{10pt}
\renewcommand{\headrulewidth}{0pt}
\renewcommand{\footrulewidth}{0pt}

\usepackage{titlesec}
%\titleformat{\part}[display]
    %{\filcenter}
    %{\Large\MakeUppercase{\partname} \thepart}
    %{12pt}
    %{\Large\bfseries}{}
    
\titleformat{\chapter}[display]
    {\filcenter}
    {\MakeUppercase{\chaptertitlename}\thechapter}
    {8pt}
    {\bfseries}{}
 
\titleformat{\section}
    {\normalsize\bfseries\center}
    {\thesection}
    {0.5em}{}
 
\titleformat{\subsection}
    {\normalsize\bfseries}
    {\thesubsection}
    {1em}{}
    
% Отступ заголовков
\titlespacing*{\section}{0pt}{*0}{*0}

% Нумерация без точек после цифры
\renewcommand{\theenumi}{\arabic{enumi}}
\renewcommand{\theenumii}{\arabic{enumi}.\arabic{enumii}}
\renewcommand\labelenumi{\theenumi}
\renewcommand\labelenumii{\theenumii}
 
% Настройка вертикальных и горизонтальных отступов
%\titlespacing*{\chapter}{0pt}{-30pt}{8pt}
\titlespacing*{\section}{\parindent}{*0}{*0}
\titlespacing*{\subsection}{\parindent}{*4}{*4}

% Работа с переносами
\pretolerance=7000
\doublehyphendemerits=1000000
\hyphenpenalty=1000
\uchyph=1 % Запрет переноса в заголовках

\sloppy

% Подавление висячих строк
\clubpenalty=10000
\widowpenalty=10000

\parindent=0.55cm % Отступ
\singlespacing % Межстрочный интервал
\frenchspacing % Одинарные пробелы

% Команда для задания УДК
\newcommand{\UDK}[1]{\noindent\textbf{УДК}~#1}

\newcommand{\annatation}[1]{\footnotesize{\textit{#1}}\normalsize\vspace{1em}}
\newcommand{\keywords}[1]{\footnotesize{\textbf{Ключевые слова:}~#1}\normalsize\vspace{1em}}

% Счетчик статей
\newcounter{topic}
\addtocounter{topic}{0} % set them to some other numbers than 0
\newcommand{\topic}[1]{
	\section*{#1} 
	% Обнуление внутренних счетчиков для каждой статьи
	\setcounter{equation}{0}
	\setcounter{table}{0}
	\setcounter{figure}{0}
	\setcounter{footnote}{0}
}

% Команда для указания авторов
\newcommand{\authors}[1]{
	\vspace{1em} \noindent \emph{#1}
}

% Команда для указания адреса
\newcommand{\addres}[1]{#1}

% Команда для указания email
\newcommand{\email}[1]{
	e-mail: \emph{#1}
}

% Команда для добавления заголовка в содержание
\newcommand{\topiccontent}[2] {
	\addtocounter{topic}{1}
	\addcontentsline{toc}{section}{
		%\arabic{part}.\arabic{topic}.
		\emph{#2} \newline #1
	}
}

% Нумерация талиц и рисунков в тексте
\newcommand{\btref}[1]{\ref{#1}}

% Нумерация формул
\renewcommand{\theequation}{\arabic{equation}}

% Форматирование подписей таблиц и рисунков
\DeclareCaptionLabelFormat{gostfigure}{Рисунок #2}
\DeclareCaptionLabelFormat{gosttable}{Таблица #2}
\DeclareCaptionLabelSeparator{gost}{~---~}
\captionsetup{labelsep=gost}
\captionsetup[figure]{labelformat=gostfigure, justification=centering, margin=0.55cm}
\captionsetup[table]{labelformat=gosttable, margin={0.55cm, 0cm}, singlelinecheck=false, justification=justified}

\makeatletter

% Настройка отображения списка ли тературы
\newcommand{\re}{\vspace{-0.5em}}
\renewcommand*{\@biblabel}[1]{ #1.\hfill }

% Верхний колонтитул для страниц
\renewcommand{\@oddhead}{\hbox to 108mm{\hrulefill\raisebox{1.8mm}{\underline{\strut{\small\bfseries\slshape Безопасность в техносфере 2021}}}}}

% Колонтитулы для страниц без нумерации !!!!!!!!!! не работает (съезжает нумерация) !!!!!!!!!!!!!!
\fancypagestyle{btPageStyleR}{
	\fancyhf{}
	\fancyhead[R]{\hbox to 108mm{\hrulefill\raisebox{1.8mm}{\underline{\strut{\small\bfseries\slshape Безопасность в техносфере 2021}}}}}
}

\fancypagestyle{btPageStyleLI}{
	\fancyhf{}
	\fancyhead[L]{\hbox to 108mm{\raisebox{1.8mm}{\underline{\strut{\footnotesize\bfseries\slshape Раздел I}}}\hrulefill}}
}

% Настройка отображения содержания
\renewcommand{\@dotsep}{3}
\renewcommand{\l@part}{\@dottedtocline{1}{0em}{1.25em}}
\renewcommand{\l@section}{\@dottedtocline{2}{1.25em}{1.75em}}
\renewcommand{\l@subsection}{\@dottedtocline{3}{2.75em}{2.6em}}

\makeatletter

% Оформление заголовков
\newcommand{\btparagraph}[1]{\vspace{1em}\noindent \textbf{#1}\hfill\vspace{1em}}

% Команда, которая задает год
\renewcommand{\year}{2021}

% Команда, которая вставляет в статью текст об участии в гранте
\newcommand{\btgrant}{Работа поддержана грантом Минобрнауки №RFMEFI57414X0038 в рамках реализации ФЦП <<Исследования и разработки по приоритетным направлениям развития научно-технологического комплекса России на 2014 -- 2020 годы>>}

\newcommand{\argmin}{\arg\!\min}

